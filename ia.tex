% Sweave(file = "ia.Rnw")
\documentclass[a4paper]{article}
\usepackage{geometry}
\geometry{a4paper}
\pagestyle{plain}
%\usepackage{Sweave}
\usepackage[latin1]{inputenc}  %encodage du fichier source
\usepackage[francais,spanish,english]{babel}
\usepackage[section]{placeins} %The placeins package provides the command \FloatBarrier
\usepackage{rotating}
\usepackage[table,usenames,dvipsnames]{xcolor}
\usepackage{hyperref} %gestion des hyperliens

\usepackage{graphicx}
\graphicspath{ {./image/} }
\DeclareGraphicsExtensions{.jpg,.pdf,.mps,.png}
\usepackage{float}
\usepackage[table,usenames,dvipsnames]{xcolor}
\usepackage{authblk} % plusieurs auteurs et affiliations
\usepackage[table,usenames,dvipsnames]{xcolor}
\usepackage{setspace} %set space between lines
\definecolor{deeppink}{RGB}{255,20,147}
\usepackage{color}

%h	Place the float here, i.e., approximately at the same point it occurs in the source text (however, not exactly at the spot)
%t	Position at the top of the page.
%b	Position at the bottom of the page.
%p	Put on a special page for floats only.
%!	Override internal parameters LaTeX uses for determining "good" float positions.
%H	Places the float at precisely the location in the LaTeX code. Requires the float package,[1] e.g., \usepackage{float}. This is somewhat equivalent to h!.
\hypersetup{
     backref=true,    %permet d'ajouter des liens dans...
     pagebackref=true,%...les bibliographies
     hyperindex=true, %ajoute des liens dans les index.
     colorlinks=true, %colorise les liens
     breaklinks=true, %permet le retour ? la ligne dans les liens trop longs
     urlcolor= blue,  %couleur des hyperliens
     linkcolor= blue, %couleur des liens internes
     bookmarks=true,  %cr?? des signets pour Acrobat
     bookmarksopen=true,            %si les signets Acrobat sont cr??s,
                                    %les afficher completement.
     pdftitle={Relation IA indice d'abondance}, %informations apparaissant dans
     pdfauthor={Ga�lle Germis, C�dric Briand},     %dans les
     % informations du document
     pdfsubject={anguille},          %sous Acrobat
     pdfkeywords={anguille}{epa}
}
% header on the first page
%\usepackage{fancyhdr}
%\pagestyle{fancyplain}
%\renewcommand{\headrulewidth}{0.4pt} 
%\fancyhead[R]{\includegraphics[height=1cm]{LogoAZTI.jpg}}
%\fancyhead[L]{\includegraphics[height=1cm]{LogoIAV.jpg}}
%\fancyfoot[C]{Not yet a draft version}

\title{\Huge{Relation Indice d'abondance densit�}}
\author[1]{Ga�lle Germis}
\author[1]{C�dric Briand}


\affil[1]{BGM, 9 rue Louis Kerautret Botmel, 35067 Rennes}
\affil[2]{IAV, bd de bretagne, 56130 La Roche Bernard}
\usepackage[round]{natbib}
\bibliographystyle{plainnat}

\begin{document}
\maketitle
%ctrl'
%\thispagestyle{plain}%using fancyhdr, this option reverts to page numbers in the center of the footer
\begin{center}
\includegraphics[width=15cm]{ang.jpg} 
\end{center}
\newpage

\begin{abstract}
Texto del abstract
\end{abstract}


\section{R�sultats principaux}



\input{./data/lm1.tex}
voir Figure \ref{plotia} et tableau \ref{lm1}.
%=============================================
\begin{figure}[htbp]
\centering
\includegraphics[width=0.6\textwidth]{plotia.pdf}% width= permet de
% changer le zoom pour afficher le graph
		\caption[Relation ia-densit�]{
        Relation IA-densit�
	} 
	%entre []=toto court et entre {}=toto long pour la l�gende et/ou la liste des figures					
\label{plotia}
\end{figure}
%=============================================
%=============================================
\begin{figure}[htbp]
\centering
\includegraphics[width=0.9\textwidth]{plottailsta.pdf}% width= permet de
% changer le zoom pour afficher le graph
		\caption[Structure en taille]{
        Structure en taille compar�e des anguilles pr�lev�es par EPA et par
        p�che�lectrique deux passages au h�ron } 
	%entre []=toto court et entre {}=toto long pour la l�gende et/ou la liste des figures					
\label{plotia}
\end{figure}
%=============================================
L'efficacit� moindre sur la p�che IA est li�e � 
\begin{itemize}
  \item En 2013 d�bit faible sur beaucoup de stations ce qui diminue
  l'efficacit� du martin, alors que le h�ron � deux passages est moins affect�.
  \item Habituellement les stations IA anguille sont d�finies sur des habitats
  de type radier rapide. Les stations IAV sont majoritairement des habitats
  peu courants, et des radiers courts, les IA d�bordent sur des plats peu
  favorables en terme d'habitat.
  \item La p�che une semaine apr�s le passage au h�ron pourrait avoir induit un
  �chappement des anguilles de la station.
\end{itemize}  

Pas de biais li� au d�parement.
Pas de modification de l'efficacit� en fonction de la taille des anguilles
captur�es p=0.21.
  
\input{./data/strt.tex}
\section{A faire}
Trouver les anguilles dans la station et hors de la station
\end{document}
